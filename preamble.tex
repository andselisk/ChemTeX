%%%%%%%%%%%%%%%%%%%%%%%%%%%%%%%%%%%%%%%%%%%%%%%%%%%%%%%%%%%%%%%%%%%%%%%%%%%%%%%%
%                                   PACKAGES                                   %
%%%%%%%%%%%%%%%%%%%%%%%%%%%%%%%%%%%%%%%%%%%%%%%%%%%%%%%%%%%%%%%%%%%%%%%%%%%%%%%%

% Resolve conflict between <chemmacros> package and command for hyperbolic cosine
\let\Ch\ch    % Save the command for the hyperbolic cosine
\let\ch\relax % Undefine \ch

\usepackage[T1]{fontenc}

\usepackage[english]{babel}

\usepackage{geometry}
\geometry{
  left   = 3.00cm,
  right  = 1.75cm,
  top    = 2.00cm,
  bottom = 2.00cm
}

\sloppy

\usepackage{amsmath}

\usepackage{amsfonts}

\usepackage{amssymb}

\usepackage{epigraph}

\usepackage{tabularx}

\usepackage{graphicx}

\usepackage{xcolor}

\usepackage{float}

\usepackage{tikz}
  \usetikzlibrary{
    decorations.markings
  }
  \tikzset{>=latex}

\usepackage{tikz-cd}

\usepackage{pgfplots}
\pgfplotsset{
%  samples = 200,
}

\usepackage{subcaption}

\usepackage{upgreek}

\usepackage{chemfig}

\usepackage{chemmacros}
\chemsetup{
  modules                   = all,
  formula                   = chemformula, % chemformula|mhchem|chemist|chemfig
  chemformula/frac-style    = math,        % math|xfrac|nicefrac
  chemformula/lewis-default = pair,        % .|:|||o|single|pair|pair (dotted)|pair (line)|empty
  chemformula/circled       = formal,      % formal|all|none
  chemformula/circletype    = chem,        % chem|math
  acid-base/p-style         = upright,     % italics|slanted|upright
  charges/circled           = formal,      % formal|all|none
  charges/circletype        = chem,        % chem|math
  nomenclature/iupac        = auto,        % auto|restricted|strict
  particles/elpair          = dots,        % dots|dash|false
  phases/pos                = side,        % side|sub
  isotopes/format           = super,       % super|side
  polymers/delimiters       = [],
  reactions/before-tag      = R,
  reactions/tag-open        = [,
  reactions/tag-close       = ],
  redox/parse               = true,        % true|false
  redox/roman               = true,        % true|false
  redox/pos                 = super,       % top|super|side
  greek                     = upgreek,
}

\usepackage{siunitx}
\sisetup{
  detect-family = true,
  detect-all
}

% \usepackage{microtype}

% \usepackage{mmap}

\usepackage[
  backend    = biber,
%  style      = chem-acs,
  sorting    = none,
  firstinits = true,
  doi        = true,
  isbn       = true,
  url        = true,
]{biblatex}
\addbibresource{../../Library/Chemistry/references-chem.bib}

\usepackage[
  autostyle,
  english = american
  ]{csquotes}

\graphicspath{{graphics/}}

\usepackage[colorlinks]{hyperref}
\makeatletter
\hypersetup{
  pdftitle   = {\@title},
  pdfauthor  = {\@author},
  colorlinks = true,
  unicode    = true,
  citecolor  = red,
  filecolor  = black,
  linkcolor  = blue,
  urlcolor   = blue
}
\makeatother

\usepackage[xindy,toc]{glossaries}
\makeglossaries

\listfiles % Show packages info in log-file

% Add new picture
\newcommand{\addimg}[3]{
  \begin{figure}
    \begin{center}
      \includegraphics[scale=#2]{#1}
    \end{center}
    \caption{#3}
  \end{figure}
}

%%%%%%%%%%%%%%%%%%%%%%%%%%%%%%%%%%%%%%%%%%%%%%%%%%%%%%%%%%%%%%%%%%%%%%%%%%%%%%%%
%                               NEW COMMANDS                                   %
%%%%%%%%%%%%%%%%%%%%%%%%%%%%%%%%%%%%%%%%%%%%%%%%%%%%%%%%%%%%%%%%%%%%%%%%%%%%%%%%

% Add supplementary information
\newcommand{\beginsupplement}{
  \setcounter{table}{0}
  \renewcommand{\thetable}{S\arabic{table}}
  \setcounter{figure}{0}
  \renewcommand{\thefigure}{S\arabic{figure}}
}

% List of accented words
\newcommand{\cterm}[1]{\textit{#1}}          % Define chemical term
\newcommand{\mterm}[1]{\textit{\textsf{#1}}} % Define machinery term

% List of common chmicals abbreviations
\newcommand{\nPr}{\textit{\textsuperscript{n}}Pr}
\newcommand{\iPr}{\textit{\textsuperscript{i}}Pr}
\newcommand{\nBu}{\textit{\textsuperscript{n}}Bu}

%%%%%%%%%%%%%%%%%%%%%%%%%%%%%%%%%%%%%%%%%%%%%%%%%%%%%%%%%%%%%%%%%%%%%%%%%%%%%%%%
%                                    MACROS                                    %
%%%%%%%%%%%%%%%%%%%%%%%%%%%%%%%%%%%%%%%%%%%%%%%%%%%%%%%%%%%%%%%%%%%%%%%%%%%%%%%%

%Drawing brackets in polymers, copied from p. 43  http://mirror.ox.ac.uk/sites/ctan.org/macros/generic/chemfig/chemfig_doc_en.pdf
\newcommand\setpolymerdelim[2]{\def\delimleft{#1}\def\delimright{#2}}
\def\makebraces[#1,#2]#3#4#5{%
  \edef\delimhalfdim{\the\dimexpr(#1+#2)/2}%
  \edef\delimvshift{\the\dimexpr(#1-#2)/2}%
  \chemmove{%
    \node[at=(#4),yshift=(\delimvshift)]
    {$\left\delimleft\vrule height\delimhalfdim depth\delimhalfdim
      width0pt\right.$};%
    \node[at=(#5),yshift=(\delimvshift)]
    {$\left.\vrule height\delimhalfdim depth\delimhalfdim
      width0pt\right\delimright_{\rlap{$\scriptstyle#3$}}$};}}
\setpolymerdelim()

